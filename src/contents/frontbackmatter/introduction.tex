\chapter{Introduction}\label{ch:introduction}
    Quantum mechanics has traditionally been formulated on the foundation of Hermitian operators, ensuring real eigenvalues and the conservation of probability. This Hermiticity requirement has long been considered essential for the physical viability of quantum theories. However, in recent years, Carl Bender and his collaborators have pioneered the study of quantum systems governed by non-Hermitian but rather \PT-symmetric Hamiltonians, where \mcP\ denotes Parity Inversion and \mcT\ denotes Time Reversal tranformations. Remarkably, such systems can exhibit entirely real spectra and unitary time evolution which conserves probability under certain conditions, allowing for the building of a physical Theory of non-Hermitian Quantum Mechanics.

    This thesis is organized as follows. The first chapter provides a review of the traditional Hermitian Quantum Mechanics, focusing on the role and properties of Hermitian operators \cite{Shankar2012-kg,Sakurai2020-pu,Bernardini1993-iy}. Subsequently, the concepts of Parity Inversion (\mcP), Time Reversal (\mcT), and their combination (\PT) are introduced, starting from the classical case crafted upon the Lorentz Group \cite{bender2024} and then expanding on their quantum mechanical version, introducing the operators \hP, \hT\ and \hPT\ along with a discussion of \PT\ Symmetry in Quantum Mechanics \cite{Bender2005,bender2024}. The following chapter addresses the problem of completeness in \PT-symmetric systems and shows how the concept of \CPT\ Symmetry naturally arises when attempting to restore completeness and orthogonality \cite{Bender2007,bender2024}. Finally, applications of \PT\ Symmetric Quantum Mechanics are explored, including a finite-dimensional Hamiltonian and deformations of the Harmonic Oscillator \cite{Bender1998}, illustrating the physical implications and mathematical structure of these new quantum systems.