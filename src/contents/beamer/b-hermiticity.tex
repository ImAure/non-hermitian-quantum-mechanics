\section{Operatori Hermitiani}
\begin{frame}{Osservabili fisiche}
    Nella Meccanica Quantistica standard, gli stato fisico $\ket{\psi}$ è un vettore in uno spazio di Hilbert \mcH. Le quantità osservabili sono associate ad operatori Hermitiani su \mcH.

    \pause
    Il processo di misura di un'osservabile \hA :
    \begin{enumerate}[label=\mybullet]
        \pause
        \item Distrugge lo stato fisico $\ket{\psi}$;
        \pause
        \item Fa collassare il sistema nell'autostato $\ket{\phi_n}$ di \hA\ associato all'autovalore $\lambda_n$.
        \pause
        \item Restituisce come misura l'autovalore $\lambda_n$ di \hA;
    \end{enumerate}
    \pause
    La misura è un fenomeno aleatorio, la probabilità di osservare l'autovalore $\lambda_n$ è
    \begin{equation*}
        P\pqty{\lambda_n} = \abs{\braket{\phi_n}{\psi}}^2
    \end{equation*}
    \refer{A.N. \etal, PRB (2019)}
\end{frame}

\begin{frame}
    Come mai è importante che gli operatori siano Hermitiani?
    \begin{enumerate}[label=\mybullet]
        \pause
        \item Spettro reale:
            \begin{equation*}
                \hA\!\ket{\phi_n} = \lambda_n\!\ket{\phi_n}
                \quad\implies\quad
                \lambda_n \in \bbR
                \mycomma
            \end{equation*}
            essenziale affinché le misure abbiano senso fisico;
        \pause
        \item Autospazi ortogonali:
            \begin{equation*}
                \braket{\phi_m}{\phi_n} = \tensor{\delta}{_m_n}\mycomma
            \end{equation*}
            le possibili misure sono mutualmente esclusive;
        \pause
        \item Generano trasformazioni di simmetria:
            \begin{equation*}
                \hU = \myexp{i\hA}
                \quad\implies\quad
                \hU~\text{unitario}
                \myperiod
            \end{equation*}
    \end{enumerate}
\end{frame}

