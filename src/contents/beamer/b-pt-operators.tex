\section{L'operatore \hPT}
\begin{frame}{Operatori \hP\ e \hT}
    Introduciamo gli operatori quantistici \hP\ e \hT.
    \begin{enumerate}[label=\mybullet]
        \pause
        \item Inversione di parità \hP : inverte posizione e impulso e il sistema appare specchiato:
            \begin{equation*}
                \hP\hvr\hP^{-1} = -\hvr
                \mycomma
                \qquad
                \hP\hvp\hP^{-1} = -\hvp
                \mysemicolon
            \end{equation*}
        \pause
        \item Inversione temporale \hT : inverte l'impulso ma non la posizione, il sistema appare come se si evolvesse a ritroso:
            \begin{equation*}
                \hT\hvr\hT^{-1} = \hvr
                \mycomma
                \qquad
                \hT\hvp\hT^{-1} = -\hvp
                \myperiod
            \end{equation*}
    \end{enumerate}
    \pause
    Si pone inoltre $\comm*{\hP}{\hT}\! = 0$.
\end{frame}

\begin{frame}{Proprietà di \hP\ e \hT}
    Applicando \hP\ e \hT\ parentesi di commutazione canoniche
    \begin{equation*}
        \comm*{\hx_j}{\hp_k}\! = i\hbar\tensor{\delta}{_j_k}
    \end{equation*}
    devono conservarsi. Da questo si trova che:
    \begin{enumerate}[label=\mybullet]
        \pause
        \item \hP\ è lineare, Hermitiano, unitario e di riflesisone, ovvero
            \begin{equation*}
                \hP = \hP^\dag = \hP^{-1}
                \mycomma\qquad
                \hP^2 = \hidM
                \myperiod
            \end{equation*}
        I suoi autovalori sono $\pm1$.
        \pause
        \item \hT\ è \emph{antilineare}, ovvero $\hT\lambda\!\ket{\psi} = \lambda^*\hT\!\ket{\psi}$, e \emph{anitunitario}. Posto $\ket*{\psi'} = \hT\!\ket{\psi}$ e $\ket*{\phi'} = \hT\!\ket{\phi}$ si ha 
            \begin{equation*}
                \braket*{\phi'}{\psi'} = \braket*{\phi}{\psi}^*
                \quad\implies\quad 
                \abs*{\!\braket*{\phi'}{\psi'}} = \abs*{\!\braket*{\phi}{\psi}}
                \mycomma
            \end{equation*}
        quindi conserva la probabilità.
    \end{enumerate}
\end{frame}

\begin{frame}{Operatore \hPT}
    L'operatore \hPT\ è la composizione di \hP\ e \hT. Di conseguenza è antiunitario e per sistemi senza spin è tale che $\bqty*{\hPT}\!^2 = \hidM$.

    \pause
    Possiamo imporre che abbia autovalori uguali a $1$:
    \begin{equation*}
        \hPT\!\ket{\psi} = \lambda\!\ket{\psi}
        \quad\implies\quad
        \lambda = e^{i\theta}
        \mysemicolon
    \end{equation*}
    
    Ridefiniamo $\ket*{\psi'} = e^{i\theta/2}\!\ket{\psi}$ e otteniamo
    \begin{equation*}
        \hPT e^{i\theta/2}\!\ket{\psi}
        =\pause e^{-i\theta/2}\hPT\!\ket{\psi}
        =\pause e^{-i\theta/2}e^{i\theta}\!\ket{\psi}
        =\pause e^{i\theta/2}\!\ket{\psi}
        \myperiod
    \end{equation*}
    \pause
    da cui $$\hPT\!\ket*{\psi'} = \ket*{\psi'}$$
\end{frame}

% \begin{frame}{Il gruppo di Lorentz}
%     Le trasformazioni di Lorentz sono quelle trasformazioni che lasciano invariato il 4-intervallo:
%     \begin{equation*}
%         \Delta s^2 = \tensor{x}{_\mu}\tensor{x}{^\mu} = c^2t^2 - x^2 - y^2 - z^2
%         \myperiod
%     \end{equation*}
%     \pause
%     I cambi di segno di una o più delle coordinate, come
%     \begin{enumerate}[label=\mybullet]
%         \item $ct \to -ct$
%         \item $x \to -x$
%         \item $\vb{r} \to -\vb{r}$, con $\vb{r} \equiv \pqty{x,y,z}$
%     \end{enumerate}
%     conservano il $4$-intervallo, quindi sono ancora trasformazioni di Lorentz.
% \end{frame}
% 
% \begin{frame}{Trasformazioni \mcP\ e \mcT}
%     Introduciamo due trasformazioni:
%     \begin{enumerate}[label=\mybullet]
%         \pause
%         \item Inversione di parità, che inverte le coordinate spaziali:
%             \begin{equation*}
%                 \func{\mcP}{\pqty{ct,\vb{r}}}{\pqty{ct,-\vb{r}}}
%             \end{equation*}
%         \pause
%         \item Inversione temporale, che inverte il tempo
%             \begin{equation*}
%                 \func{\mcT}{\pqty{ct,\vb{r}}}{\pqty{-ct,\vb{r}}}
%             \end{equation*}
%             \pause L'effetto dell'Inversione temporale in realtà è quello di invertire la direzione del moto.
%     \end{enumerate}
% \end{frame}
% 
% \begin{frame}{Connessione del gruppo di Lorentz}
%     Le trasformazioni di Lorentz formano un gruppo e sono identificate dal parametro $\vb*{\beta} = \vb{v}/c$.
% 
%     \pause
%     Il gruppo di Lorentz si divide in quattro componenti connesse:
%     \begin{enumerate}[label=\mybullet]
%         \pause
%         \item Il sottogruppo contentente l'identità;
%         \pause
%         \item Il sottogruppo dell'identità composto con \mcP;
%         \item Il sottogruppo dell'identità composto con \mcT;
%         \item Il sottogruppo dell'identità composto con \mcP\mcT;
%     \end{enumerate}
%     \pause
%     Queste quattro componenti sono sconnesse tra loro.
%     
%         
% \end{frame}
% 
% \begin{frame}{Connessione del gruppo di Lorentz}
%     Le componenti possono essere ridotte a due passando ai numeri complessi.
%     \begin{figure}
%         \begin{tikzpicture}
%             \draw (1,1) -- (1,2);
%         \end{tikzpicture}
%         \qquad\pause
%         \begin{tikzpicture}
%             \draw (1,1) -- (2,1);
%         \end{tikzpicture}
%     \end{figure}
%     \pause
%     La componente dell'identità si connette alla componente \PT.
%     \pause
%     Le trasformazioni \PT\ potrebbero quindi avere un ruolo privilegiato.
% \end{frame}

% \begin{frame}{Passaggio agli operatori quantistici}
%     Introduciamo gli operatori quantistici \hP\ e \hT, analoghi dei classici \mcP\ e \mcT :
%     \begin{enumerate}[label=\mybullet]
%         \pause
%         \item L'inversione di parità inverte posizione e impulso, il sistema appare come allo specchio:
%             \begin{equation*}
%                 \hP\hvr\hP^{-1} = -\hvr
%                 \mycomma
%                 \qquad
%                 \hP\hvp\hP^{-1} = -\hvp
%                 \mysemicolon
%             \end{equation*}
%         \pause
%         \item L'inversione temporale inverte l'impulso ma non la posizione, il sistema appare come se si evolvesse a ritroso:
%             \begin{equation*}
%                 \hT\hvr\hT^{-1} = \hvr
%                 \mycomma
%                 \qquad
%                 \hT\hvp\hT^{-1} = -\hvp
%                 \myperiod
%             \end{equation*}
%     \end{enumerate}
%     \pause
%     Si pone inoltre $\comm*{\hP}{\hT}\! = 0$.
% \end{frame}

% \begin{frame}{Linearità e antilinearità}
%     Imponendo che le trasformazioni \hP\ e \hT\ lascino invariate le relazioni di commutazione canoniche
%     \begin{equation*}
%         \comm*{\tensor{\hx}{_j}}{\tensor{\hp}{_k}}\! = i\hbar\tensor{\delta}{_j_k}
%         \mycomma
%     \end{equation*}
%     si trova che 
%     \begin{enumerate}[label=\mybullet]
%         \item \hP\ deve essere lineare, ovvero $\forall\ket{\psi},\ket{\phi}\in\mcH$, $\forall\lambda,\mu\in\bbC$
%             $$
%                 \hP\pqty{\lambda\!\ket{\psi} + \mu\!\ket{\phi}}
%                 = \lambda\hP\!\ket{\psi} + \mu\hP\!\ket{\phi}
%             $$
%         \item \hT\ deve essere \emph{antilineare}, ovvero
%             $$
%                 \hT\pqty{\lambda\!\ket{\psi} + \mu\!\ket{\phi}}
%                 = \lambda^*\hT\!\ket{\psi} + \mu^*\hT\!\ket{\phi}
%             $$
%     \end{enumerate}
% \end{frame}
% 
% \begin{frame}{Proprietà di \hP}
%     Si può inoltre dimostrare che \hP :
%     \begin{enumerate}[label=\mybullet]
%         \pause
%         \item È Hermitiano $$\hP = \hP^\dag\mysemicolon$$
%         \pause
%         \item È unitario $$\hP^{-1} = \hP^\dag\mysemicolon$$
%         \pause
%         \item Il suo quadrato è l'identità $$\hP^2 = \hidM\myperiod$$
%     \end{enumerate}
%     \pause
%     Di conseguenza, gli autovalori di \hP\ sono $\pm1$: $$\hP\!\ket{\pm} = \pm\!\ket{\pm}.$$
% \end{frame}
% 
% \begin{frame}{Proprietà di \hT}
%     Per \hT\ il discorso è più delicato. È un operatore \emph{antiunitario}
%     Se trasformo due vettori
%     $$\ket*{\psi'} = \hT\!\ket{\psi}\quad\text{e}\quad\ket*{\phi'} = \hT\!\ket{\phi}\mycomma$$
%     \pause
%     e ne faccio il prodotto scalare, ottengo il complesso coniugato:
%     $$\braket*{\phi'}{\psi'} = \braket{\phi}{\psi}^*\myperiod$$
%     \pause
%     L'operatore \hT\ non conserva il prodotto scalare ma conserva il suo modulo
%     $\abs*{\!\braket*{\phi'}{\psi'}}\! = \abs{\braket{\phi}{\psi}}$,
%     essenziale per conservare la probabilità.
% 
%     \pause 
%     Infine $\hT^2 = \hidM$ vale solo per i sistemi con spin intero.
% \end{frame}

% \begin{frame}{Operatore \hPT}
%     L'operatore \hPT\ è la composizione di \hP\ e \hT. Di conseguenza è antiunitario e per sistemi senza spin è tale che $\bqty*{\hPT}\!^2 = \hidM$.
% 
%     \pause
%     Possiamo imporre che abbia autovalori uguali a $1$:
%     \begin{equation*}
%         \hPT\!\ket{\psi} = \lambda\!\ket{\psi}
%         \quad\implies\quad
%         \lambda = e^{i\theta}
%         \mysemicolon
%     \end{equation*}
%     
%     Ridefiniamo $\ket*{\psi'} = e^{i\theta/2}\!\ket{\psi}$ e otteniamo
%     \begin{equation*}
%         \hPT e^{i\theta/2}\!\ket{\psi}
%         =\pause e^{-i\theta/2}\hPT\!\ket{\psi}
%         =\pause e^{-i\theta/2}e^{i\theta}\!\ket{\psi}
%         =\pause e^{i\theta/2}\!\ket{\psi}
%         \myperiod
%     \end{equation*}
%     \pause
%     da cui $$\hPT\!\ket*{\psi'} = \ket*{\psi'}$$
% \end{frame}