\section{Operatori \hP, \hT\ e \hPT}
\begin{frame}{Il gruppo di Lorentz}
    Le trasformazioni di Lorentz sono quelle trasformazioni che lasciano invariato il 4-intervallo:
    \begin{equation*}
        \Delta s^2 = \tensor{x}{_\mu}\tensor{x}{^\mu} = c^2t^2 - x^2 - y^2 - z^2
        \myperiod
    \end{equation*}
    \pause
    I cambi di segno di una o più delle coordinate, come
    \begin{enumerate}[label=\mybullet]
        \item $ct \to -ct$
        \item $x \to -x$
        \item $\vb{r} \to -\vb{r}$, con $\vb{r} \equiv \pqty{x,y,z}$
    \end{enumerate}
    conservano il $4$-intervallo, quindi sono ancora trasformazioni di Lorentz.
\end{frame}

\begin{frame}{Trasformazioni \mcP\ e \mcT}
    Introduciamo due trasformazioni:
    \begin{enumerate}[label=\mybullet]
        \pause
        \item Inversione di parità, che inverte le coordinate spaziali:
            \begin{equation*}
                \func{\mcP}{\pqty{ct,\vb{r}}}{\pqty{ct,-\vb{r}}}
            \end{equation*}
        \pause
        \item Inversione temporale, che inverte il tempo
            \begin{equation*}
                \func{\mcT}{\pqty{ct,\vb{r}}}{\pqty{-ct,\vb{r}}}
            \end{equation*}
    \end{enumerate}
\end{frame}

\begin{frame}{Connessione del gruppo di Lorentz}
    Le trasformazioni di Lorentz formano un gruppo
    
\end{frame}