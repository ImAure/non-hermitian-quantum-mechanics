\section{Applicazioni}
\begin{frame}{Deformazione dell'oscillatore armonico}
    Consideriamo l'Hamiltoniana adimensionata
    \begin{equation*}
        \hH = \hp^2 + \hx^2\pause\pqty{i\hx}^\varepsilon
        \mycomma
    \end{equation*}
    La deformazione non è Hermitiana ma è \PT-invariante per $\varepsilon > -1$.

    \begin{enumerate}[label=\mybullet]
        \pause
        \item Per $\varepsilon \geq 0$: Simmetria non rotta, autovalori reali;
        \pause
        \item Per $-1 < \varepsilon < 0$: Simmetria rotta, un numero finito di autovalori sono reali, il resto sono complessi;
        \pause
        \item Per $\varepsilon \to -1$ lo spettro diverge. 
    \end{enumerate}
\end{frame}

\begin{frame}{Potenziale quartico capovolto}
    Un caso interessante si ha per $\varepsilon = 2$:
    \begin{equation*}
        \hH = \hp^2 - \hx^4
    \end{equation*}
    \pause
    Attraverso metodi di analisi complessa si trova che lo spettro coincide con quello di $\hH = \hp^2 + \hx^4$.
    \begin{enumerate}[label=\mybullet]
        \pause
        \item  Le soluzioni sono stati legati;
        \pause
        \item La particella percorre orbite ellittiche nel piano complesso;
        \pause
        \item La posizione media è l'origine. 
    \end{enumerate}
    Ricorda il potenziale quartico di auto-interazione in teoria dei campi.
\end{frame}