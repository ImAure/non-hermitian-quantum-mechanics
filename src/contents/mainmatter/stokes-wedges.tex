\chapter[Boundary Conditions in \PT\ Symmetric Quantum Mechanics][Boundary Conditions]{Boundary Conditions in \PT\ Symmetric Quantum Mechanics}\label{a:stokes}
    \section*{Stokes wedges}
        In the body of this thesis we briefly mentioned the topic of boundary conditions when we stated that the wavefunction must always be square Lebesgue-integrable in order to belong in the Hilbert space $L^2$. This condition must be taken into account when solving the Schr\"odinger equation, as the differential equation itself only describes the wavefunction locally depending on $\pqty{\vb{r},t}$. The integrability of the wavefunction constitutes a stronger \emph{global} condition when compared to Hermiticity, \PT-symmetry and other \emph{local} conditions, which may or may not hold depending on $\pqty{\vb{r},t}$.

        To face the problem of boundary conditions we often make asymptotic approximations in order to study the property of the wavefunction as $\abs{x}\to+\infty$. However, when dealing with \PT\ Symmetry we may find eigenfunctions that do not vanish fast enough when restricted to the real line, but do so when extended to certain regions of the complex plane. For this reason we often need to deal with asymptotic approximations in the complex plane. While this is a simple task in \bbR, it becomes non trivial when dealing with complex variables, as we may find regions of the complex plane where the function behaves very differently.

        The regions in which the approximations are valid depend on the function in consideration. The function
        \begin{align*}
            \sinh\!\pqty{z} = \frac{e^z - e^{-z}}{2}
        \end{align*}
        behaves very differently when $\Re z > 0$ and $\Re z < 0$, as it diverges exponentially to $+\infty$ and $-\infty$ respectively. The regions of the complex plane in which the asymptotic approximations are well-defined and coherent with the behaviour of the function are usually determined by the angular interval in which $\arg z$ may vary. To specify the region in which the approximation is valid we also need to specify the limit $z \to z_0$, where $z_0$ may be infinity. Such a region is called a \emph{Stokes wedge} and its boundary is called a \emph{Stokes line}. For $\sinh z$ we have two Stokes wedges, one for $-\pi/2 < \arg z < \pi/2$ and one for $\pi/2 < \arg z < 3\pi/2$, where the function asymptotically behaves as $e^z/2$ and $-e^{-z}/2$ respectively. 

    \section*{More on the quartic upside-down potential}
        The reason we are interested in Stokes wedges is that they allow us to define the boundary conditions of the wavefunction in a more general way. We may ask that the wavefunction vanishes in a certain Stokes wedge rather than on the real line, thus simplifying the solution of the Schr\"odinger equation for \PT\ Symmetric Quantum Mechanics. Let us consider the multiplicative deformation of the Harmonic Oscillator Hamiltonian we studied in \chref{ch:examples}
        \begin{equation*}
            \hH = \hp^2 + x^2\pqty{i x}^\varepsilon
            \mycomma \qquad
            \varepsilon \in \bbR 
            \myperiod
        \end{equation*}
        In this case, the continuous deformation parameter $\varepsilon$ also continuously deforms the Stokes wedges in which the wavefunction must vanish. The Stokes wedges must also be symmetric with respect to the imaginary axis in order to preserve \PT\ Symmetry. When $\varepsilon = 0$, the Stokes wedges are centered around the real axis with an opening angle of $\pi/2$, therefore including the real line itself. As $\varepsilon$ increases the Stokes wedges rotate downwards about the origin and their opening becomes smaller.

        A noteworthy case is again when $\varepsilon = 2$, corresponding to the upside-down quartic potential, where the Stokes lines partially lie on the real axis. This does not allow the wavefunction to be defined on the real line, as it would not vanish at infinity, but we may still define the wavefunction in the appropriate Stokes wedges and then take the limit as $\Im z \to 0^-$. The divergence of the wavefunction on the real line means that the particle cannot be found on the real line as it would travel to infinity in a \emph{finite} amount of time. This time can be evaluated as
        \begin{equation*}
            t_0 = \int_{0}^{+\infty} \frac{\dd{x}}{2\sqrt{E + x^4}} < +\infty
            \myperiod
        \end{equation*}
        One may wonder where the particle would be after this finite time, and the answer can only be found in the complex plane. By extending the classical equations to complex numbers we find that the particle travels in closed trajectories embedded in the complex plane, with a period which is exactly twice the time $t_0$. This may be interpreted as the particle traveling to infinity and then coming back in the same amount of time. The trajectories are also elliptical and symmetric with respect to both axes, meaning that the particle's average position is in fact the origin. This is coherent with the real positive and discrete spectrum of the Hamiltonian which describes a bound state \cite{bender2024}.