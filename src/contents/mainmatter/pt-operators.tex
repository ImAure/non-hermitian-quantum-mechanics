\chapter{Parity Inversion and Time Reversal Operators}\label{ch:pt-operators}
    \section{Spacetime transformations}\label{s:spacetime-transformations}
        The \emph{Parity Inversion} and \emph{Time Reversal} operators are best understood by looking into the context of special relativity first.

        The Lorentz group is the group of all $4\times4$ real matrices that transform a $4$-vector $\tensor{x}{^\mu} = \pqty{ct,x,y,z} = \pqty{ct,\vb{r}}$ into another $4$-vector $\tensor{{x'}}{^\mu}$ while preserving the scalar $4$-interval\footnote{The signature $\pqty{+,-,-,-}$ for the Minkowski spacetime is implied.}
        \begin{equation}
            \label{eq:4-interval}
            \Delta s^2
            = \tensor{x}{_\mu} \tensor{x}{^\mu}
            = c^2t^2 - x^2 - y^2 - z^2
            \myperiod
        \end{equation}
        By this definition, it is clear that transformations that change the sign of one or more of the components are allowed. The most general Lorentz group, which is often referred to as $O\pqty{1,3}$, is not connected, as it is impossible to continuously deform a transformation that includes a sign flip into the identity transformation.

        We will only focus on four of the connected components of this group. In order to describe them, we shall introduce the following symmetry transformations. Let \mcP\ be the map 
        \begin{equation}
            \label{eq:lorentz-P}
            \func{\mcP}{\pqty{ct,\vb{r}}}{\pqty{ct,-\vb{r}}}
        \end{equation}
        and, similarly, let \mcT\ be the map
        \begin{equation}
            \label{eq:lorentz-T}
            \func{\mcT}{\pqty{ct,\vb{r}}}{\pqty{-ct,\vb{r}}}
            \mycomma
        \end{equation}
        let $\PT \equiv \mcP\circ\mcT$ be the composition of \mcP\ and \mcT.
        The \mcP\ and \mcT\ maps are usually named \emph{Parity inversion} and \emph{Time reversal}, they are Lorentz transformations and matrix representation is simply given by the two diagonal matrices
        \begin{equation*}
            \mcP = \pmqty{\dmat{1,-1,-1,-1}}
            \qand
            \mcT = \pmqty{\dmat{-1,1,1,1}}
            \mycomma
        \end{equation*}
        It is also evident that $\mcP^2 = \mcT^2 = \idM$ and $\PT = \mcT\mcP = -\idM$.
        
        We can now determine the four components of the subgroup \cite{bender2024}:
        \begin{enumerate}[label = \mybullet]
            \item The \emph{proper orthochronous Lorentz group}, also known as the identity component since it is the connected subgroup of $O\pqty{1,3}$ containing the identity transformation. This group of transformations does not flip the sign of neither the time coordinate nor the one of the spatial coordinates.
            \item The three distinct products of the proper orthochronous Lorentz group with \mcP, \mcT\ and \PT\ respectively, containing the transformations whose effect involves either flipping the sign of the time coordinate, the sign of the spatial coordinates or both simultaneously.
        \end{enumerate}
        None of the last three sets of transformations constitute a subgroup of themselves: the only actual subgroup is the identity component as it is the only one containing the identity transformation.
        
        It can be shown \cite{bender2024} that the number of disconnected components in the Lorentz group can be brought down to two if we extend the group to the complex numbers. When we allow the coefficients to be complex, the ``\mcP\ component'' and the ``\mcT\ component'' of the group become connected and the same can be said of the proper orthochronous subgroup and the ``\PT\ component''. This suggests that the \PT\ symmetry could play a more fundamental role as a symmetry of nature compared to \mcP\ or \mcT\ separately.
    
    \section{Definition and properties of \hP\ and \hT}\label{s:def-of-P-and-T}
        Moving on from the preliminary definition of the classical \mcP, \mcT\ and \PT\ operators in the context of Lorentz transformations, we now introduce their quantum analogous along with some of their properties.

        In the context of non-relativistic Quantum Mechanics we expect the quantum operators to behave in a way similar to their classical counterparts, as the correspondence principle states. In particular, if we limit our attention to the one dimensional case for the sake of simplicity, we would like the parity inversion operator to map $\hx$ to $-\hx$ and $\hp$ to $-\hp$, while the time reversal operator should leave $\hx$ unaltered and map $\hp$ to its opposite, $-\hp$. The reasoning behind this definition is quite intuitive: flipping the spatial coordinates will just flip both the position and the direction of motion of a particle; flipping the time coordinate will not affect the position of the particle but it will appear as though it is moving backwards.

        In order to satisfy these requirements, the quantum \hP\ and \hT\ operators are conventionally defined by their action on \hx\ and \hp\ as follows:
        \begin{align}
            \label{eq:quantum-P}
            &\hP\hx\hP^{-1} = -\hx
            \mycomma
            &\hP\hp\hP^{-1} = -\hp
            \mycomma
            \\
            \label{eq:quantum-T}
            &\hT\hx\hT^{-1} = \hx
            \mycomma
            &\hT\hp\hT^{-1} = -\hp
            \myperiod
        \end{align}
        Since the inversion of parity and the reversal of time are independent actions, the assumption\footnote{The convention $\comm*{\hA}{\togglehat{B}} = \hA\togglehat{B} - \togglehat{B}\hA$ is implied.}
        \begin{equation}
            \label{eq:PT-commutator}
            \comm*{\hP}{\hT} = \hzero 
        \end{equation}
        is legit.

        Having defined the operators, let us go over the property of linearity. Since we are trying to introduce some kind of symmetry, we ask that the laws of Physics be unchanged under these symmetry transformations, down to the fundamental level. Hence, the canonical commutation relation $\comm*{\hx}{\hp} = i \hbar$ should hold even after transforming the operators through \hP\ and \hT. By using the earlier definitions, it is easy to show that this is indeed true if some assumptions on the operators are made; for \hP\ we have
        \begin{align*}
            i \hbar 
            = \comm*{\hx}{\hp}\! 
            = \comm*{-\hx}{-\hp} \!
            = \bigl[\hP \hx \hP^{-1},\hP \hp \hP^{-1}\bigr] 
            = \hP \comm*{\hx}{\hp}\! \hP^{-1} 
            = \hP i \hbar \hP^{-1} 
        \end{align*}
        which can only be true if
        \begin{equation}
            \label{eq:PiP}
            \hP i \hP^{-1} = i
            \myperiod
        \end{equation}
        On a similar fashion, for \hT we get
        \begin{align*}
            i \hbar
            = \comm*{\hx}{\hp}\! 
            = \bigl[\hT \hx \hT^{-1},-\hT \hp \hT^{-1}\bigr] 
            = \hT \comm*{\hx}{-\hp}\! \hT^{-1}
            = \hT \pqty{-i \hbar} \hT^{-1}
            \mycomma
        \end{align*}
        however, since the commutator must be preserved, we need
        \begin{equation}
            \label{eq:TiT}
            \hT i \hT^{-1} = -i
            \mycomma
        \end{equation}
        which means that the \hT\ operator \emph{is not linear}. We can therefore say for any $z \in \bbC$ 
        \begin{alignat}{3}
            \label{eq:P-linear}
            \hP z \hP^{-1} &= z        
            &\quad\iff\quad 
            \hP z &= z \hP
            \mycomma
            \\
            \label{eq:T-antilinear}
            \hT z \hT^{-1} &= z^* 
            &\quad\iff\quad 
            \hT z &= z^* \hT
            \mycomma
        \end{alignat}
        where $z^*$ is the hermitian conjugate %\footnote{Being a number, the hermitian conjugate $z^\dag$ and the complex conjugate $z^*$ are in fact the same thing. The $\dag$ notation is used because \hP\ and \hT\ are defined by their action on operators: $\hT z \hT^{-1}$ actually means $\hT \pqty{z\togglehat{\idM}} \hT^{-1} = z^*\togglehat{\idM} = \pqty*{z \togglehat{\idM}}^\dag$.}
        of $z$. In this context, \hT\ is said to be \emph{antilinear}, which means that when we apply \hT\ to a linear combination of states with complex coefficients we can distribute the sum and bring the coefficients to the left of the operator by directly applying \eqref{eq:T-antilinear}. In symbols
        \begin{equation*}
            \hT\pqty{c_1\!\ket{\psi} + c_2\!\ket{\phi}}
            = c_1^* \hT\!\ket{\psi} + c_2^* \hT\!\ket{\phi}
            \myperiod
        \end{equation*}

        \subsection{Eigenvalues of \hP, Hermiticity, unitarity}
            We can show that \hP\ is both Hermitian and unitary. Starting from $\hvr\!\ket{\vb{r}} = \vb{r}\!\ket{\vb{r}}$ and by looking at \eqref{eq:lorentz-P}, it is easy to convince ourselves that applying the Parity Inversion operator on $\ket{\vb{r}}$ will yield $\ket{-\vb{r}}$.\footnote{This can be achieved by multiplying \eqref{eq:r-eigenvalues} on the left by \hP\ and introducing the identity $\hidM = \hP^{-1}\hP$, getting $\hP\hvr\hP^{-1}\hP\!\ket{\vb{r}} = \hP\vb{r}\!\ket{\vb{r}}$. It easily follows that $\hP\!\ket{\vb{r}}$ is an eigenstate of \hvr\ with eigenvalue $-\vb{r}$.} Through \eqref{eq:integral-identity} we have
            \begin{equation*}
                \mel{\phi}{\hP}{\psi}
                = \int\nolimits_{\bbR^3}\bra{\phi}\!\hP\!\ket{\vb{r}}\!\!\bra{\vb{r}}\ket{\psi}\dd[3]{\vb{r}}
                = \int\nolimits_{\bbR^3}\phi^*\pqty{-\vb{r}}\psi\pqty{\vb{r}}\dd[3]{\vb{r}}
                \myperiod
            \end{equation*}
            Since we are integrating over all of $\bbR^3$ we are allowed to rename the variable of integration from $x$ to $-x$, and by considering the Hermitian conjugate of the identity $\hP\!\ket{\vb{r}} = \ket{-\vb{r}}$, which is $\bra{\vb{r}}\!\hP^\dag = \bra{-\vb{r}}$, we get
            \begin{equation*}
                \int\nolimits_{\bbR^3}\phi^*\pqty{\vb{r}}\psi\pqty{-\vb{r}}\dd[3]{\vb{r}}
                = \int\nolimits_{\bbR^3}\bra{\phi}\ket{\vb{r}}\!\!\bra{\vb{r}}\!\hP^\dag\!\ket{\psi}\dd[3]{\vb{r}}
                = \mel{\phi}{\hP^\dag}{\psi}
                \mycomma
            \end{equation*}
            from which we see that $\hP^\dag = \hP$. From the fact that $\hP^2\!\ket{\vb{r}} = \ket{\vb{r}}$ we also get $\hP^2 = \hidM$, implying unitarity. To sum up,
            \begin{equation}
                \label{eq:P-properties}
                \hP
                = \hP^\dag
                = \hP^{-1}
                \qand
                \hP^2
                = \togglehat{\idM}
                \myperiod
            \end{equation}

            Being linear, Hermitian and unitary, due to the Spectral Theorem the \hP\ operator possesses real unimodular eigenvalues. If we agree on naming $\ket{\pm}$ the eigenstate of \hP\ with eigenvalue $\pm 1$, we can write
            \begin{equation}
                \label{eq:eigenvalue-problem-P}
                \hP\!\ket{\pm}
                = \pm\!\ket{\pm}\myperiod
            \end{equation}
        
        \subsection{Eigenvalues of \hT, antiunitarity}
            Going back to \hT, having already shown that it is antilinear, we are not really allowed to talk about Hermiticity nor unitarity. We can however show that \hT\ preserves the modulus of scalar products and, consequently, the norm. This is remarkable, since the probability of finding the state $\ket{\psi}$ in $\ket{\phi}$ is $\abs{\braket{\phi}{\psi}}^2$, and this quantity only depends on the modulus of the \emph{probability amplitude} $\braket{\phi}{\psi}$.
            
            Because of \hT's antilinearity we have $\mel{\vb{r}}{\hT}{\psi} = \psi^*\pqty{\vb{r}}$. If we evaluate the inner product between two transformed states $\hT\!\ket{\phi}$ and $\hT\!\ket{\psi}$, we find
            \begin{equation*}
                \bra{\phi}\!\hT^\dag\hT\!\ket{\psi}
                = \int\nolimits_{\bbR^3} \phi\pqty{\vb{r}}\psi^*\pqty{\vb{r}}\dd[3]{\vb{r}}
                = \bra{\psi}\ket{\phi}
                = \bra{\phi}\ket{\psi}^*
                \mycomma
            \end{equation*}
            which means that the original probability amplitude $\braket{\phi}{\psi}$ and the transformed one $\bra{\phi}\!\hT^\dag\hT\!\ket{\psi}$ share the same modulus. An operator that conserves the modulus of the inner product but turns the inner product itself via complex conjugation is said to be \emph{antiunitary}. We proved the antiunitarity of \hT\ by requiring that the canonical commutation relations be conserved under 
            \hT\ transformations, however the two properties are equivalent: one can define \hT\ as antiunitary and show that the canonical commutation relations are in fact conserved.
            
            Let's now figure out how the eigenvalue problem translates to antiunitary operators. An antiunitary operator can be expressed \cite{Sakurai2020-pu} as the product of an unitary linear part $\hU$ and antilinear part $\hK$ which solely deals with the complex conjugation, hence
            \begin{equation}
                \label{eq:antiunitary-operator}
                \hT = \hU\!\hK
                \myperiod
            \end{equation}
            The properties of an antiunitary operator may vary depending on the system in consideration. In particular, we may show that $\hT^2$ is not necessarily the identity operator, unlike the classical case where $\mcT^2 = \idM$. In the case of a physical state with spin $s = 1/2$, we may express the unitary part $\hU$ as
            \begin{equation*}
                \hU = \myexp{-i \frac{\hSy}{\hbar} \pi}
                \mycomma
            \end{equation*}
            where the choice of phase $\pi$ is made in order to invert the direction of spin \hvS, as we would expect from the Time Reversal transformation, not differently from how it transforms \hvp. In this case
            \begin{equation*}
                \hT^2 = \myexp{-i \frac{\hSy}{\hbar} \pi}\! \hK \myexp{-i \frac{\hSy}{\hbar} \pi}\! \hK
            \end{equation*}
            and, in a basis where $\hSz$ is diagonal, we know that $\hSy$ is purely imaginary in  the form
            \begin{equation*}
                \hSy
                = \frac{\hbar}{2} \togglehat{\sigma}_y
                = \frac{\hbar}{2} \!\pmqty{\admat{-i,i}}
                \mycomma
            \end{equation*}
            thus, the exponent in \hU\ is actually real and the \hK\ operator has no effect on it. It follows that
            \begin{equation*}
                \hT^2
                = \myexp{-i \frac{\hSy}{\hbar} 2\pi}
                = \myexp{-i \togglehat{\sigma}_y \pi}
                \mycomma
            \end{equation*}
            showing that the double application of \hT\ corresponds to a rotation of an angle $\pi$ about the $y$ axis. As a consequence the eigenstates of \hSz, which we shall call $\ket{\pm s}$, will be flipped by $\hT^2$:
            \begin{equation*}
                \hT^2 \!\ket{\pm s} = \mp\! \ket{\pm s}
                \quad\iff\quad
                \hT^2 = -\togglehat{\idM}
                \myperiod
            \end{equation*}
            A similar argument is valid for any half-integer value of $s$. In the situation where spin is an integer number $\pqty{s = 0,1,\ldots}$, the double application of \hK\ will yield the exponential of an integer multiple of $2\pi$. Under these circumstances we have
            \begin{equation*}
                \hT^2\! \ket{\pm} = \pm\!\ket{\pm}
                \quad\iff\quad
                \hT^2 = +\togglehat{\idM}
                \myperiod
            \end{equation*}

            The eigenvalue problem only admits solutions in the case of integer spin or, more generally, in all cases where $\hT^2 = +\togglehat{\idM}$. It can be proved by considering the eigenvalue problem for \hT
            \begin{equation}
                \label{eq:eigenvalue-problem-T}
                \hT\! \ket{\psi} = \lambda\! \ket{\psi}
                \myperiod
            \end{equation}
            Since \hT\ is antiunitary, the modulus of $\lambda$ must be unity, as
            \begin{equation*}
                \bra{\psi}\!\hT^\dag \hT \!\ket{\psi}
                = \bra{\psi} \!\lambda^* \lambda\!\ket{\psi}
                = \abs{\lambda}^2 \!\braket{\psi}{\psi}
                \quad\implies\quad
                \abs{\lambda}^2 = 1
                \mycomma
            \end{equation*}
            however, by multiplying equation \eqref{eq:eigenvalue-problem-T} by \hT\ on the left we get
            \begin{equation*}
                \hT^2\! \ket{\psi}
                = \hT \lambda\! \ket{\psi}
                = \lambda^* \hT\! \ket{\psi}
                = \lambda^* \lambda\! \ket{\psi}
                = \abs{\lambda}^2\! \ket{\psi}
                \mycomma
            \end{equation*}
            where it is evident that solutions may only exist when $\hT^2 = +\togglehat{\idM}$. In the opposite case the equation would reduce to $\abs{\lambda}^2 = -1$ which is absurd. This result will be important later, as we discuss the existence of eigenstates for the \hPT\ operator. For the sake of simplicity, we are only going to consider spinless systems, so that $\hT^2 = \togglehat{\idM}$ holds.

    \section{\PT\ Symmetry}\label{s:pt-symmetry}
        It is now time to introduce the composite operator \hPT, the concept of \PT\ transformations and \PT\ Symmetry, which will be of major interest throughout the rest of this thesis.

        The \hPT\ operator is defined as the composition of Parity Inversion and Time Reversal. Since by definition \eqref{eq:PT-commutator} \hP\ and \hT\ commute, \hPT\ and $\hT\hP$ are the same. We can easily determine the way \hPT\ transforms \hx\ and \hp\ by relying on \eqref{eq:quantum-P} and \eqref{eq:quantum-T}:
        \begin{align}
            \label{eq:quantum-PT-x}
            \hPT\hx\bqty*{\hPT}\!^{-1}
            &= -\hx
            \mycomma
            \\
            \label{eq:quantum-PT-p}
            \hPT\hp\bqty*{\hPT}\!^{-1}
            &= \hp
            \myperiod
        \end{align}
        \hPT\ is antilinear just like \hT\ and therefore cannot be Hermitian nor unitary; it is however antiunitary and its properties can be derived from the ones of \hP\ and \hT. In particular, from the known fact that $\hP^2 = \hidM$ and the assumption $\hT^2 = \hidM$, we also get that $\bqty*{\hPT}^2 = \hidM$. 

        We can extend the definition of \PT\ tranformation to any operator \hA\ that can be expressed as a function of \hx\ and \hp, namely
        \begin{equation}
            \label{eq:pt-transform}
            \hA' = \PTtransform{\hA}
            \myperiod
        \end{equation}
        We say that an operator \hA\ is \emph{\PT-symmetric} if $\PTtransform{\hA} = \hA$. The condition of \PT-symmetry can be expressed in a cleaner way via a little manipulation:
        \begin{equation*}
            \PTtransform{\hA} = \hA
            \quad\iff\quad
            \hPT\!\hA = \hA\hPT
            \quad\iff\quad
            \hPT\!\hA - \hA\hPT = \hzero
            \mycomma
        \end{equation*}
        which means that condition of \PT-symmetry for \hA\ is equivalent to
        \begin{equation}
            \label{eq:pt-symmetric-operator}
            \comm*{\hPT}{\hA}\! = \hzero
        \end{equation}
        The most trivial example of a \PT-symmetric operator is the momentum operator \hp, right from the definition \eqref{eq:quantum-PT-p}. Since $\comm*{\hPT}{\hp}\! = \hzero$, any regular function of \hp\ is \PT-symmetric. The position operator \hx\ is not \PT-symmetric itself but its power $\hx^2$ is:
        \begin{align*}
            \bigl[\hPT,\hx^2\bigr]
            &= \hx \bigl[\hPT,\hx\bigr] + \bigl[\hPT,\hx\bigr] \hx \\
            &= -2\hx^2\hPT + 2\hx^2\hPT \\
            &= \hzero 
            \mycomma
        \end{align*}
        using the fact that $\hPT \hx = -\hx \hPT$ and $\comm*{\hPT}{\hx}\! = -2\hx\hPT$.

        Since \hPT\ is antiunitary, the existence of its eigenvalues is non-trivial as thoroughfully discussed at \Sref{s:def-of-P-and-T} while talking about \hT. Let's consider the eigenvalue problem
        \begin{equation}
            \label{eq:eigenvalue-problem-PT}
            \hPT\! \ket{\psi} = \lambda\! \ket{\psi}
            \mycomma
        \end{equation}
        by relying of the assumption that $\hT^2 = \hidM$ and the known fact that $\hP^2 = \hidM$, we see that $\bqty*{\hPT}\!^2 = \togglehat{\idM}$ too. Therefore the eigenvalue problem \eqref{eq:eigenvalue-problem-PT} actually admits solutions with unimodular eigenvalues. Without loss of generality we are allowed to choose $\lambda = 1$, since any physical state $\ket{\psi}$ is defined up to a pure constant phase \cite{Bender2005}. This can be shown by operating the substitution $\ket{\psi} \rightarrow e^{i\theta / 2}\!\ket{\psi}$, where $e^{i\theta} = \lambda$; the eigenvalue problem \eqref{eq:eigenvalue-problem-PT} then becomes
        \begin{equation*}
            \hPT e^{i\frac{\theta}{2}}\! \ket{\psi}
            = e^{-i\frac{\theta}{2}} \hPT\! \ket{\psi}
            = e^{-i\frac{\theta}{2}} e^{i\theta}\! \ket{\psi}
            = e^{i\frac{\theta}{2}}\! \ket{\psi}
            \mycomma
        \end{equation*}
        showing that $e^{i\theta / 2}\! \ket{\psi}$ indeed is an eigenstate for \hPT\ with eigenvalue $1$.

        We are now ready to introduce the concept of \PT\ Symmetry. Let's consider a linear operator \hA\ such that $\comm*{\hPT}{\hA}\! = \hzero$. It may be tempting to assume that, since \hA\ and \hPT\ commute, a set of simultaneous eigenstates for both operators exists. However this is only true for \emph{linear} operators with a null commutator, whereas \hPT\ is \emph{antilinear}. For this reason the previous statement is false \cite{bender2024} and we will have to check by hand, case by case, whether a set of simultaneous eigenstates exists. We will now show that when such a set can be found the eigenvalues of \hA\ are real \cite{bender2024}. We start from the eigenvalue problem for \hA,
        \begin{equation}
            \label{eq:eigenvalue-problem-A}
            \hA\! \ket{\psi} = \lambda\! \ket{\psi}
            \mycomma
        \end{equation}
        and by assuming that $\ket{\psi}$ is an eigenstate with eigenvalue $1$ for \hPT\ as well. If we multiply equation \eqref{eq:eigenvalue-problem-A} on the left by \hPT, we obtain
        \begin{equation*}
            \hPT\! \hA\! \ket{\psi} = \hPT \lambda\! \ket{\psi}
            \quad\iff\quad
            \hA \hPT\! \ket{\psi} = \lambda^* \hPT\! \ket{\psi}
            \quad\iff\quad
            \hA\! \ket{\psi} = \lambda^*\! \ket{\psi}
            \myperiod
        \end{equation*}
        By comparing this result with \eqref{eq:eigenvalue-problem-A} we deduce that $\lambda^* = \lambda \in \bbR$.

        When the \PT-symmetric operator in question is the Hamiltonian \hH\ and a set of simultaneous eigenstates for \hH\ and \hPT\ exists, we say that the physical system described by the Hamiltonian \hH\ is \emph{\PT\ Symmetric} and that \emph{\PT\ Symmetry is unbroken}. Please note that the \PT-symmetry of the Hamiltonian is not enough by itself to guarantee that \PT\ Symmetry be unbroken: \PT\ Symmetry of the system and \PT-symmetry of the Hamiltonian are not equivalent. This is an interesting property since it allows us to extend the theory of Quantum Mechanics to a more general set of Hamiltonians that need not be Hermitian anymore, since when \PT\ Symmetry is not broken the energy measurements will be real numbers. This generalization does however come at a cost, which we shall discuss in the following chapter.