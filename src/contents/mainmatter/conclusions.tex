\chapter{Conclusions}\label{ch:conclusions}
    In this thesis we saw how \PT\ Symmetric Quantum Mechanics can be a consistent extension of standard Quantum Mechanics, being able to describe physical systems that do not break \PT\ Symmetry. One of the most important results is that \PT-symmetric Hamiltonians can be seen as one possible natural extension of Hermitian Hamiltonians, along with the proof that a \PT-symmetric Hamiltonian can be mapped to a Hermitian isospectral one via a similarity transformation. This means that we can interpret \PT\ Symmetric Quantum Machanics not as a new theory, but rather as a different representation of the well established Quantum Mechanics.
    
    This connection is also highlighted by the strong formal analogies between the two theories that we constructed in \Sref{ch:pt-symmetric-hamiltonians} and underlined in Table \ref{tab:lunica}. By replacing the usual Hermitian conjugation with \CPT\ conjugation, we recovered some fundamental conditions, like completeness and unitarity, that are virtually identical to Hermitian Quantum Mechanics. The presence of these analogies, along to the similarity of the two theories, suggests that \PT\ Symmetry could be one of the fundamental symmetries of nature---being itself an expression of Lorentz Invariance, as we saw in \Sref{s:spacetime-transformations}---and the extension of Quantum Mechanics to \PT-Symmetric Hamiltonians is not a purely mathematical construction but rather a physical one.

    Moreover we found out how imposing \PT\ Symmetry naturally brought \CPT\ Symmetry into the picture, reinforcing the fundamental role that symmetries play in our reality. The \CPT\ Symmetry suggests a connection to Quantum Field Theory, where the \CPT\ theorem states that any Lorentz-invariant local quantum field theory with a Hermitian Hamiltonian must be invariant under the combined operations of Charge Conjugation \hC, Parity Inversion \hP, and Time Reversal \hT. Even though the \CPT\ Theorem is not directly applicable to \PT\ Symmetric Quantum Mechanics and the role of \hC\ is not as clear as the one of \hP\ and \hT\ outside of the realm of Quantum Field Theory.

    Finally we saw how \PT\ Symmetric Quantum Mechanics can be applied to some physical systems, verifying the theoretical predictions and the consistency of the theory. The most notable example is the quartic upside-down potential, at which we hinted in \Sref{s:deformed-harmonic-oscillator}, which correctly describes a real physical system, suggesting a connection between \PT\ Symmetric Quantum Mechanics and Quantum Field Theory.

    There are still many open questions regarding this topic, however this also means there are many opportunities for future research, starting fromt the study of more deformations of well known Hermitian systems, to the search of novel systems that can only be described in the formalism of \PT\ Symmetric Quantum Mechanics. The connection to Quantum Field Theory is also relevant since it may lead to a better and deeper understanding of the fundamental symmetries of nature.