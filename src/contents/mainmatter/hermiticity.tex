\chapter{Hermitian Operators in Quantum Mechanics}\label{ch:hermiticity}
    \section{Postulates of Hermitian Quantum Mechanics}
        Hermitian operators play a fundamental role in Quantum Mechanics since its postulates directly bind them to physical observables. Let's go over these postulates \cite{Shankar2012-kg} and their implications \cite{Shankar2012-kg,Sakurai2020-pu} succinctly:
        \begin{enumerate}[label=\Roman*.] 
            \item The state of the particle is represented by a vector $\ket{\psi\pqty{t}}$ in a Hilbert space.
            \item The classical independent variables $x$ and $p$ are represented by Hermitian operators $\hx$ and $\hp$ with the following matrix elements in the eigenbasis $\ket{x}$ of $\hx$:
            \begin{align*}
                \mel*{x}{\hx}{x'} &= x\delta\pqty*{x-x'} \mycomma \\
                \mel*{x}{\hp}{x'} &= -i\hbar\pdv{x}\delta\pqty*{x-x'} \myperiod
            \end{align*}
            The operators corresponding to dependent variables $\omega\pqty{x,p}$ are given by the same function of the operators $\togglehat{\Omega}\pqty*{\hx,\hp}$.
            \item If the particle is in the state $\ket{\psi}$, measurement of the observable $\togglehat{\Omega}$ will yield one of the eigenvalues $\omega_n$ of $\togglehat{\Omega}$ with probability $\abs{\braket{\omega_n}{\psi}}^2$. The state of the system will change from $\ket{\psi}$ to $\ket{\omega_n}$ as a result of the measurement.
            \item The state vector $\ket{\psi\pqty{t}}$ evolves according to Schr\"odinger's wave equation
            \begin{equation}
                i\hbar\pdv{t}\!\ket{\psi\pqty{t}} = \hH\!\ket{\psi\pqty{t}}
                \mycomma
                \label{eq:schrodinger}
            \end{equation}
            where $\hH = \hH\pqty*{\hx,\hp}$ is the quantum Hamiltonian operator of the system.
        \end{enumerate}

        The Schr\"odinger's wave equation describes the evolution in time of the physical state in the Hilbert space and it is of little use in its purely formal expression. A common way to make the problem easier to deal with is the introduction of the eigenstates of the position operator \hx, and its vector analogous $\hvr = \pqty{\hx,\hy,\hz}$, which allows us to translate the problem in the language of functions. We shall name $\ket{x}$ the eigenstate of \hx\ with eigenvalue $x$, in symbols
        \begin{equation}
            \label{eq:x-eigenvalues}
            \hx\!\ket{x} = x\!\ket{x}
            \mycomma
        \end{equation}
        and, with similar considerations, we can write its vector form
        \begin{equation}
            \label{eq:r-eigenvalues}
            \hvr\!\ket{\vb{r}} = \vb{r}\!\ket{\vb{r}}
            \myperiod
        \end{equation}

        \subsection{Introduction to wavefunctions}
            If a state $\ket{\psi}$ coincides with an eigenstate $\ket{\vb{r}}$, it means that the particle is infinitely localized at position $\vb{r}$. If we were to introduce a function, or rather a \emph{wavefunction}, to help us in mathematically defining the spacial distribution of a particle in space, we should rightfully expect it to be shaped like a Dirac's delta. We can in fact define the wavefunction for any physical state $\ket{\psi}$: we know from the postulates of Quantum Mechanics that the quantity $\abs{\braket{\vb{r}}{\psi}}^2$ is the proability---or rather the \emph{probability density}, since $\vb{r}$ varies continuously---of a position measurement yielding the eigenvalue $\vb{r}$. We hence define the wavefunction as
            \begin{equation}
                \label{eq:vavefunction}
                \psi\pqty{\vb{r}} = \braket{\vb{r}}{\psi}
                \mycomma
            \end{equation}
            which is a complex valued function whose modulus squared is notoriously a density of probability. For the wavefunction to be well defined, we need the total probability to be unity, namely 
            \begin{equation}
                \label{eq:normalization}
                \int\nolimits_{\bbR^3}\abs{\psi\pqty{\vb{r}}}^2 \dd[3]{\vb{r}} = 1
                \mycomma
            \end{equation}
            and, for this reason, we choose $\ket{\psi}$ and $\ket{\vb{r}}$ to be of unitary norm. This is allowed because, while $\ket{\psi}$ and $\lambda\ket{\psi}$ for any $\lambda\in\bbC$ are not generally the same vector in the Hilbert space, they are actually equivalent in representing a physical state, since, when substituted in Schr\"odinger's equation \eqref{eq:schrodinger}, the constant $\lambda$ cancels out. % By the properties of 
            Having defined the wavefunction, it is easy to see that for any eigenstate $\ket{\vb{r}_0}$ of \hvr\ the associated wavefunction would be $\psi\pqty{\vb{r}} = \braket{\vb{r}}{\vb{r}_0} = \delta^3\pqty{\vb{r} - \vb{r}_0}$.

            The wavefunction formalism is not strictly necessary, since a quantum mechanical theory could be entirely developed by the usage of operatorial formalism in the Heisenberg picture. It is however quite useful since it allows us to translate the problems from the abstract Hilbert space to the more concrete $L^2$ space of modulus squared Lebesgue-integrable functions. In this space, the scalar product between physical states can be expressed in terms of the integral
            \begin{equation}
                \label{eq:integral-inner-product}
                \braket{g}{f} = \int\nolimits_{\bbR^3}g^*\pqty{\vb{r}} f\pqty{\vb{r}} \dd[3]{\vb{r}}
                \mycomma
            \end{equation}
            where $g^*\pqty{\vb{r}}$ is the complex conjugate of $g\pqty{\vb{r}}$. This way, it is clear that condition \eqref{eq:normalization} is just the requirement that the norm of a physical state induced by inner product \eqref{eq:integral-inner-product} be unity.

            Looking at the definition \eqref{eq:vavefunction}, we can interpret the wavefunction as the \emph{continuously varying Fourier coefficients} that allow us to express the state as an integral over the set of the set $\qty{\ket{\vb{r}}}$. This translates to
            \begin{equation*}
                \ket{\psi} = \int\nolimits_{\bbR^3} \psi\pqty{\vb{r}} \ket{\vb{r}} \dd[3]{\vb{r}}
                \mycomma
            \end{equation*}
            which can be easily manipulated into
            \begin{equation*}
                \ket{\psi} = \int\nolimits_{\bbR^3} \dyad{\vb{r}}{\vb{r}} \dd[3]{\vb{r}} \ket{\psi}
                \mycomma
            \end{equation*}
            which, in turn, can be seen as
            \begin{equation}
                \label{eq:integral-identity}
                \int\nolimits_{\bbR^3} \dyad{\vb{r}}{\vb{r}} \dd[3]{\vb{r}} = \hidM
                \myperiod
            \end{equation}
            This expression is called \emph{resolution of the Identity} and it corresponds to the spectral representation of the Identity operator. The resolution of the Identity is essential in the context of completeness: if a set of states can be used to express the Identity operator, either in a continuous way\footnote{Formally speaking, the set $\qty{\ket{\vb{r}}}$ is not a subset of the Hilbert space as its vectors have infinite norm and their wavefunctions are not square-integrable. If we however interpret their wavefunction as a distribution, they can be treated as Dirac's deltas, recovering the possibility of using them to represent physical states.} like \eqref{eq:integral-identity} or in a discrete way like
            \begin{equation*}
                \sum\nolimits_n \dyad{\phi_n}{\phi_n} = \hidM
                \mycomma
            \end{equation*}
            then the set is complete and any physical state can be expressed as a linear combination of the states in the set.
        
        \subsection{The time independent Schr\"odinger Equation}
            One of the problems of main interest in Quantum Mechanics is the one of finding the eigenvalues of the Hamiltonian operator \hH\ by solving 
            \begin{equation}
                \label{eq:schrodinger-time-independent}
                \hH\!\ket{\phi_n} = E_n\!\ket{\phi_n}
                \mycomma
            \end{equation}
            the \emph{time independent Schr\"odinger Equation}.
            This is especially useful under the condition that the Hamiltonian be time independent, allowing us to search for factored solutions in space and time. When the Hamiltonian is time independent, it is easy to verify that states of the form
            \begin{equation*}
                \ket{\psi\pqty{t}} = \myexp{-i\frac{E_n}{\hbar}t}\!\ket{\phi_n}
            \end{equation*}
            are solutions to the time dependent wave equation \eqref{eq:schrodinger}. Solutions of \eqref{eq:schrodinger-time-independent} are called \emph{stationary states} because the expectation value of any observable over a stationary state is conserved. This is easy to see with a few passages, as the quantity
            \begin{align*}
                \ev*{\hA}
                &= \ev{\hA}{\psi\pqty{t}} \\
                &= \ev{\myexp{i\frac{E_n}{\hbar}t}\hA\myexp{-i\frac{E_n}{\hbar}t}}{\phi_n} \\
                &= \ev{\hA}{\phi_n}
            \end{align*}
            does not depend on time.

            Many problems can often be reduced to finding the eigenvalues and eigenfunctions of the Hamiltonian operator. This approach allows us to determine the possible energy levels of the system and the corresponding stationary states. In practice, the time independent Schr\"odinger equation is often solved in the position representation, where the eigenvalue problem becomes a differential equation for the wavefunction:
            \begin{equation}
                \label{eq:schrodinger-position}
                \bqty{-\frac{\hbar^2}{2m}\laplacian + V\pqty{\vb{r}}}\psi_n(\vb{r}) = E_n\psi_n(\vb{r})
                \myperiod
            \end{equation}
            Here, $\psi_n\pqty{\vb{r}}$ is the eigenfunction associated to the energy eigenvalue $E_n$. The set of all such eigenfunctions forms a complete basis for the Hilbert space $L^2$, allowing any physical state to be expressed as a linear combination of stationary states. This framework is fundamental for analyzing quantum systems, especially when dealing with bound states and quantized energy spectra, and for it to be applicable in the study of a real physical system it is essential that the energy eigenvalues $E_n$ be real. This is the reason why we are interested in looking into the conditions that allow an operator to have real eigenvalues, the most common being Hermiticity.

    \section{Hermitian operators and their properties}\label{s:hermitian-operators}
        Hermitian operators are so fundamental in Quantum Mechanics because of their crucial properties, the most immediate one being the reality of their eigenvalues: applying an operator to a physiscal state corresponds to a measurement of the physical quantity represented by the operator itself and the result of such measurement will be one of the eigenvalues of the operator. Given that all physical measurements are real numbers, we need the eigenvalues of any operator representinga an observable quantity to be real. Another important property of Hermitian operators, which we shall cover further on, is their connection with unitary transformations. In Quantum Mechanics, Hermitian operators can induce unitary transformations, mirroring the role of conserved dynamical variables in Classical Mechanics as generators of infinitesimal symmetry transformations.
        
        Studying the properties of operators in a finite-dimensional vector space is an easier task, however in Quantum Mechanics we mostly have to deal with infinite-dimensional Hilbert spaces, which leads to different behaviours of the operators in some edge cases where the dimensionality of the vector space cannot be ignored. Let's now go over some of the most important and useful properties of Hermitian operators and their spectra \cite{Bernardini1993-iy} starting with the finite dimensional case. We shall then see how these properties generalize to infinite-dimensional Hilbert spaces.

        \subsection{Hermitian operators in finite dimensional spaces}
            In a $n$-dimensional Hilber space, the eigenvalue problem is stated in the usual way
            \begin{equation*}
                \hA\! \ket{\psi} = \lambda\! \ket{\psi}
                \myperiod
            \end{equation*}
            The operator \hA\ is Hermitian if it equals its own adjoint, that is
            \begin{equation*}
                \hA = \hA^\dagger
                \myperiod
            \end{equation*}
            This definition is equivalent to the condition
            \begin{equation}
                \label{eq:hermiticity-inner-product}
                \mel{\phi}{\hA}{\psi}
                = \mel{\phi}{\hA^\dag}{\psi}
                = \mel{\psi}{\hA}{\psi}^*
                \mycomma
            \end{equation}
            where $\bra{\phi}\!\hA^\dag = \bqty*{\hA\!\ket{\phi}}\!^\dag$, which must hold for any pair of states $\ket{\phi}$ and $\ket{\psi}$ in the Hilbert space. We can derive some important properties of Hermitian operators, which are all grouped under the name of the \emph{Spectral Theorem for Hermitian Operators} \cite{Bernardini1993-iy}:
            \begin{enumerate}[label = \textit{\roman{enumi}}.]
                \item An Hermitian operator \hA\ over an $n$-dimensional Hilbert space possesses $n$, not necessarily distinct, real eigenvalues $\lambda_1,\ldots,\lambda_n$. Eigenvectors associated to distinct eigenvalues are orthogonal.
                \item The operator \hA\ possesses $n$ orthonormal eigenvectors $\ket{\phi_i}$.
                \item The operator \hA\ can be expressed via its \emph{spectral decomposition}
                    \begin{equation}
                        \label{eq:spectral-decomposition}
                        \hA = \sum\nolimits_i \dyad{\phi_i}{\phi_i}\lambda_i
                        \mycomma
                    \end{equation}
                    where $\dyad{\phi_i}{\phi_i}$ is the \emph{orthogonal projection} operator, which decomposes the Hilbert space into the direct sum of $n$ mutually orthogonal eigenspaces. It naturally follows that
                    \begin{equation}
                        \label{eq:completeness-of-identity}
                        \sum\nolimits_i \dyad{\phi_i}{\phi_i} = \hidM
                        \myperiod
                    \end{equation}
                    \label{p:decomposition}
                \item In the orthonormal basis $\qty{\ket{\phi_i}}$, the matrix elements of \hA\ are
                    \begin{equation*}
                        \mel{\phi_i}{\hA}{\phi_j} = \pmqty{\dmat{\lambda_1,\ddots,\lambda_n}}
                        \myperiod
                    \end{equation*}
                    The diagonal representation of \hA\ can be achieved by a similarity tranfsormation described by a unitary operator \hU.
                \item If $m \leq n$ eigenvalues are degenerate, the matrix elements of \hA\ restricted to the $m$-dimensional eigenspace of the eigenvectors associated to the degenerate eigenvalues can be further diagonalized and the eigenspace can be further decomposed in a way similar to property \ref{p:decomposition}
                \item Any polynomial of the operator \hA\ can be expressed through its spectral decomposition
                    \begin{equation}
                        \label{eq:polynomial-of-operator}
                        p\pqty*{\hA} = \sum\nolimits_i \dyad{\phi_i}{\phi_i} p\pqty{\lambda_i}
                        \myperiod
                    \end{equation}
            \end{enumerate}
            The last property can be generalized to the expression of a generic function $f\pqty*{\hA}$ by invoking the Taylor expansion of $f$, given that we define some sort of convergence criterion for operator series first. Assuming that the series converges, and through the help of the spectral decomposition \eqref{eq:spectral-decomposition} of \hA, it can be shown that
            \begin{equation}
                \label{eq:function-of-operator}
                f\pqty*{\hA} = \sum\nolimits_n c_n \hA^n = \sum\nolimits_n \dyad{\phi_n}{\phi_n} f\pqty{\lambda_n}
            \end{equation}
            for some complex coefficients $c_n$.
        \subsection{Hermitian operators and Symmetry transformations}\label{ss:unitary-transformations}
            It can be trivially proved \cite{Bernardini1993-iy} through the use of \eqref{eq:function-of-operator} that the complex exponential of an Hermitian operator is a unitary operator:
            \begin{equation*}
                \hU = \myexp{i\hA}
                \myperiod
            \end{equation*}
            Let's focus on the more interesting connection of this property with Symmetry transformations in Classical Mechanics. Let $\hA = -\varepsilon\hG$ and let us consider the Taylor expansion of \hU
            \begin{equation*}
                \hU = \hidM - i\varepsilon\hG + o\pqty{\varepsilon}
            \end{equation*}
            The requirement that \hU\ be unitary is satisfied even if we truncate the expansion at the first order, given that \hG\ is Hermitian and $\varepsilon \to 0$. We get
            \begin{equation*}
                \hU\hU^\dag = \pqty{\hidM - i\varepsilon\hG}\pqty{\hidM + i\varepsilon\hG^\dag} + o\pqty{\varepsilon} = \hidM + i\varepsilon\pqty{\hG^\dag - \hG} + o\pqty{\varepsilon}
                \mycomma
            \end{equation*}
            which can only equal \hidM\ if $\hG^\dag = \hG$. The expectation value of another observable \hB\ will, in general, change when the transformation defined by \hU\ is applied:
            \begin{align*}
                \expval*{\hB'} &= \mel{\psi}{\hU\!\hB\hU^\dag}{\psi} \\
                &= \mel{\psi}{\pqty{\hidM - i\varepsilon\hG} \hB \pqty{\hidM + i\varepsilon\hG^\dag}}{\psi} + o\pqty{\varepsilon} \\
                &= \expval*{\hB} + i\varepsilon\!\expval*{\comm*{\hB}{\hG}\!} + o\pqty{\varepsilon}
                \mycomma
            \end{align*}
            which becomes
            \begin{equation}
                \label{eq:infinitesimal-transformation-QM}
                \expval*{\hB'} = \expval*{\hB} + i\varepsilon\!\expval*{\comm*{\hB}{\hG}\!}
            \end{equation}
            if we again let $\varepsilon \to 0$. We then deduce that any observable that commutes with \hB\ can be seen as the generator of a unitary transformation that leaves $\expval*{\hB}$ unchanged. In a more symmetric phrasing, we can say that given two operators \hA\ and \hB\ such that $\comm*{\hA}{\hB} = \hzero$, each of them is the generator of a tranformation that leaves the expectation value of the other one unchanged. The relationship \eqref{eq:infinitesimal-transformation-QM} is strikingly similar to the classical relation\footnote{The convention $\acomm{f}{g} = \pdv{f}{x}\pdv{g}{p} - \pdv{f}{p}\pdv{g}{x}$ is implied.} \cite{Goldstein2005-ep}
            \begin{equation}
                \label{eq:infinitesimal-transformation-classical}
                \var{A} = A' - A = \varepsilon\acomm{A}{g}
                \mycomma
            \end{equation}
            which relates the transformation of a dynamical variable $A\pqty{x,p}$ with the Poisson bracket of $A$ with the generating function of the transformation, $g$. From Classical Mechanics \cite{Goldstein2005-ep} we know that the Hamiltonian is the generator of translations in time. If we let $\varepsilon = \dd{t}$, relation \eqref{eq:infinitesimal-transformation-classical} becomes
            \begin{equation*}
                \dot{A} = \acomm*{A}{\mcH}
                \mycomma
            \end{equation*}
            which in turn reduces to Hamilton's equations when $A = x$ and $A = p$. By the correspondence principle we can then say that if an observable \hA\ commutes with the Hamiltonian \hH :
            \begin{enumerate}[label = \mybullet]
                \item \hA\ is conserved through time, since \hH\ is the generator of a unitary transformation, namely time translation, that leaves \hA\ unchanged;
                \item \hH\ is conserved through the transformation generated by \hA, meaning that the physical system, its properties and the equations of motion will remain the same under such transformations.
            \end{enumerate}
            It is now clear how essential the requisite that any observable quantity be described by an Hermitian operator. In Quantum Mechanics this allows us to connect conserved quantities to Symmetry transformations which, being unitary, preserve the inner product and consequently conserve the probabilities of all measurement outcomes which allow us to describe non-dissipative physical systems.

        \subsection{Extension to infinite dimensional Hilbert spaces}
            The properties of Hermitian operators in infinite-dimensional Hilbert spaces are not as straightforward as in the finite-dimensional case. Some properties, such as the reality of eigenvalues and the orthogonality of eigenvectors associated with distinct eigenvalues, still hold. However, other properties may not generalize directly or as easily \cite{Bernardini1993-iy}.

            Since we cannot represent operators with matrices in infinite dimensions, the eigenvalue problem cannot be translated into finding the roots of an $n$-degree polynomial like we would usually do when evaluating $\det\!\pqty*{\hA - \lambda\hidM} = 0$. We should rather define the \emph{resolvent operator} $\hRl = \hA - \lambda\hidM$ and ask ourselves whether \hRl\ \emph{cannot} be inverted in
            \begin{equation}
                \label{eq:eigenvalue-problem-infinite}
                \hRl\! \ket{\psi} = \vb{0}
                \mycomma
            \end{equation}
            for some $\ket{\psi} \neq \vb{0}$: when $\hRl^{-1}$ does not exist it means that the eigenvalue problem is satisfied for that specific value of $\lambda$. The inability to express the operator as a matrix also means that it cannot be diagonalized in the traditional sense. We shall nonetheless refer to operators as \emph{diagonal} when we consider a complete set of their eigenvectors as a basis for the Hilbert space.

            An important distinction that arises in infinite dimensions is between \emph{bounded} and \emph{unbounded} operators. An operator $\hA$ is bounded if there exists a constant $L \geq 0$ such that for any $\ket{\psi}$
            \begin{equation*}  
                \norm*{\hA\! \ket{\psi}} \leq L \norm*{\ket{\psi}}
                \myperiod
            \end{equation*}
            Bounded operators are continuous and defined on the whole Hilbert space, and their spectral properties are considerably easier to handle. In particular, any bounded self-adjoint operator has a well-defined spectral resolution. Unfortunately, many operators of physical interest, even the most basic ones like \hx\ and \hp\ are unbounded, and therefore require a more delicate functional-analytic treatment.

            The requirement that an operator be Hermitian does not actually guarantee the widest set of properties. The requirement that best extends the properties of a finite-dimensional Hermitian operators is actually \emph{self-adjointness}, which we just mentioned above. We say that an operator \hA\ is \emph{self-adjoint} if it satisfies $\mel{\phi}{\hA}{\psi} = \mel{\psi}{\hA}{\phi}^*$ for all $\ket{\phi}$ and $\ket{\psi}$ in its domain and the domain of its Hermitian conjugate $\hA^\dag$ is the same of \hA. For many applications, it is enough to ask that the domain of $\hA^\dag$ is \emph{dense} in the domain of \hA.

            The central result that generalizes diagonalization to infinite dimensions is still called \emph{Spectral Theorem} and it states that, given a bounded self-adjoint operator \hA\ on a Hilbert space \cite{Bernardini1993-iy}:
            \begin{enumerate}[label = \textit{\roman{enumi}}.]
                \item The spectrum of \hA, $\sigma\pqty*{\hA}$, is a bounded subset of \bbR.
                \item Eigenstates associated with distinct eigenvalues are orthogonal. \label{p:orthogonality}
                \item The operator \hA\ can be expressed in its spectral decomposition through an integral over its spectrum
                    \begin{equation}
                        \label{eq:spectral-decomposition-infinite}
                        \hA = \int\nolimits_{\sigma\pqty{\hA}} \lambda \dd{E_\lambda}
                        \mycomma
                    \end{equation}
                    where
                    \begin{equation*}
                        \dd{E_\lambda} = \sum\nolimits_i \dyad{\phi_i}{\phi_i}\delta\pqty{\lambda - \lambda_i}\dd{\lambda}
                        \myperiod
                    \end{equation*}
                \item The projection operator $\dyad{\phi_i}{\phi_i}$ decomposes the Hilbert space into the direct sum of orthogonal eigenspaces.
                \item In case of degeneracy, the eigenspace associated with the degenerate eigenvalue can be further decomposed in a similar way.
                \item The integral \eqref{eq:spectral-decomposition-infinite} reduces to an infinite series similar to \eqref{eq:spectral-decomposition} in the case that the spectrum of the operator is discrete.
            \end{enumerate}

            Under the conditions of boundedness and self-adjointness, the spectral decomposition of an operator can be used to define functions of the operator. In such cases it is possible to define unitary transformations that behave in a way similar to what we stated in \Sref{ss:unitary-transformations}.