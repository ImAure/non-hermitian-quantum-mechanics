\chapter{Examples and Applications}\label{ch:examples}
    Now that we covered the key aspects and foundations of a \PT-Symmetric quantum mechanical theory, let's look at some exaples of application to better understand how the tools we introduced operate in practical scenarios.
    
    In this chapter we will explore further the $2\times2$ matrix Hamiltonian presented in \Sref{s:problem} and look at a more concrete system obtained from the deformation of the Hamiltonian of the Quantum Harmonic Oscillator.
    \section{A finite dimensional Hamiltonian}
        Let's briefely review the $2\times2$ Hamiltonian we looked at in the previous chapter, delving into the details of the construction of \hC. The \PT-Symmetric Hamiltonian is
        \begin{equation}
            \hH = E_0 \pmqty{\dmat[1]{i\gamma, -i\gamma}}
        \end{equation}
        with both $E_0,\gamma \in \bbR$. \PT\ Symmetry is unbroken when $\abs{\gamma} < 1$. The eigenvalues and the normalized eigenvectors with respect to \PT\ inner product are
        \begin{equation*}
            \lambda_\pm = \pm \alpha E_0
            \mycomma
            \qquad
            \ket{\phi_\pm} = \frac{1}{\sqrt{2\alpha}} \pmqty{1 \\ \pm\alpha - i \gamma}
        \end{equation*}
        such that
        \begin{equation*}
            \hH\ket{\phi_\pm} = \alpha \pm E_0 \ket{\phi_\pm}
            \qand
            \braket{\phi_\pm}{\phi_\pm}_{\PT} = \pm 1
            \myperiod
        \end{equation*}
        We construct the \hC\ operator starting from \eqref{eq:C-position} which in this case becomes
        \begin{equation}
            \hC = \dyad{\phi_+}{\phi_+} + \dyad{\phi_-}{\phi_-} = \frac{1}{2\alpha}\pmqty{\dmat[2]{2i\gamma, -2i\gamma}} = \frac{1}{\alpha E_0}\hH
            \mycomma
        \end{equation}
        reminding that $\bra{\phi_\pm}$ is the \PT-conjugate of $\ket{\phi_\pm}$. Properties \eqref{eq:C-H-commutator}, \eqref{eq:C-PT-commutator} and \eqref{eq:C-eigenvalues} are trivially verified.

        Let's now determine the \CPT\ inner product, which is given by \eqref{eq:CPT-inner-product} by making use of \eqref{eq:CPT-conjugation}. The \CPT\ conjugate of $\ket{\phi_\pm}$ is
        \begin{align*}
            \bra{\phi_\pm}
            &= \bqty{\ket{\phi_\pm}}^\myconj \\
            &= \bqty*{\hCPT\ket{\phi_\pm}}^\intercal \\
            &= \frac{1}{\alpha E_0}\bqty*{\hH\hPT\ket{\phi_\pm}}^\intercal \\
            &= \frac{1}{\alpha E_0}\bqty*{\pm E_0 \alpha \ket{\phi_\pm}}^\intercal \\
            &= \pm\frac{1}{\sqrt{2\alpha}}\pmqty{\pm\alpha + i\gamma & 1}
            \myperiod
        \end{align*}
        As expected, we have
        \begin{equation}
            \braket{\phi_\pm}{\phi_\pm}_{\CPT} =  \pm\frac{1}{2\alpha}\pmqty{\pm\alpha + i\gamma & 1}\pmqty{1 \\ \pm\alpha - i \gamma} = 1
            \mycomma
        \end{equation}
        and
        \begin{equation}
            \braket{\phi_-}{\phi_+}_{\CPT} = -\frac{1}{2\alpha}\pmqty{-\alpha + i\gamma & 1}\pmqty{1 \\ \alpha - i \gamma} = 0
            \myperiod
        \end{equation}
        The set $\qty{\ket{\phi_\pm}}$ is therefore complete and orthonormal and, along with the \CPT\ inner product, gives $\bbC^2$ the structure of an Hilbert space.

        We may now express the general physical state $\ket{\psi}$ as 
        \begin{equation*}
            \ket{\psi} = c_{-}\ket{\phi_{-}} + c_{+}\ket{\phi_+}
        \end{equation*}
        for some $c_\pm\in\bbC$. The time evolved state is therefore
        \begin{align*}
            \ket{\psi\pqty{t}} 
            &= \myexp{-i\frac{\hH}{\hbar}t}\bqty*{c_{-}\ket{\phi_{-}} + c_{+}\ket{\phi_{+}}}\\
            &= c_{-}\myexp{i\omega t} \ket{\phi_{-}} + c_{+}\myexp{-i \omega t} \ket{\phi_{+}}
            \mycomma
        \end{align*}
        with $\omega = E_0 \alpha / \hbar$. It is possible to verify that this state is indeed a solution of the time dependent Schr\"odinger's equation \eqref{eq:schrodinger}.

        Finally, let's evaluate the similarity transformation that yields the Hermitian isospectral Hamiltonian \hF. We recall from \eqref{eq:C-exponential} that
        \begin{equation*}
            \myexp{\hQ} = \hC\hP
            \myperiod
        \end{equation*}
        Let $\hM = \hC\hP$ and $\hS = \myexp{-\hQ/2}$ so that
        \begin{equation}
            \label{eq:S-M-function}
            \hS = \frac{1}{\sqrt{\hM}}
            \myperiod
        \end{equation}
        \hM\ is easy to calculate from the definition of \hC\ and \hP; after we calculate \hM\ we may diagonalize it, evaluate function \eqref{eq:S-M-function} and then do the inverse transformation to find the correct expression for \hS. The operator \hM\ takes the matrix form
        \begin{equation*}
            \hM = \frac{1}{\alpha}\pmqty{\admat[1]{i\gamma, -i\gamma}}
            \mycomma
        \end{equation*}
        its eigenvalues and normalized eigenvectors are
        \begin{equation*}
            \mu_{\pm} = \frac{1}{\alpha}\pqty{1 \mp \gamma}
            \mycomma
            \qquad
            \ket{\mu_\pm} = \frac{1}{\sqrt{2}}\pmqty{1 \\ \pm i}
            \myperiod
        \end{equation*}
        If we let
        \begin{equation*}
            \hU = \frac{1}{\sqrt{2}}\pmqty{1 & -i \\ 1 & i}
            \mycomma
        \end{equation*}
        we can find the diagonal expression of \hM\ via the similarity transformation
        \begin{equation}
            \hM_d = \hU \hM \hU^\dag
            \mycomma
        \end{equation}
        which equals
        \begin{equation*}
            \hM_d = \pmqty{\dmat{\mu_{+}, \mu_{-}}}
            \myperiod
        \end{equation*}
        To evaluate \hS, we find its diagonal form by applying \eqref{eq:S-M-function} to the diagonal $\hM_d$ matrix, which yields
        \begin{equation*}
            \hS_d = \pmqty{\dmat{1/\sqrt{\mu_{+}}, 1/\sqrt{\mu_{-}}}}
            \mycomma
        \end{equation*}
        and then do the inverse transformation $\hS = \hU^\dag \hS_d \hU$ to get
        \begin{align*}
            \hS &= \frac{1}{2}\pmqty{1 & 1 \\ i & -i}  \pmqty{\dmat[0]{1/\sqrt{\mu_{+}}, 1/\sqrt{\mu_{-}}}} \pmqty{1 & -i \\ 1 & i} \\
            &= \frac{1}{2\sqrt{\mu_{+}\mu_{-}}} \pmqty{\admat[\sqrt{\mu_{+}} + \sqrt{\mu_{-}}]{i\pqty{\sqrt{\mu_{+}} - \sqrt{\mu_{-}}} , -i\pqty{\sqrt{\mu_{+}} - \sqrt{\mu_{-}}}}}
            \myperiod
        \end{align*}
        Via an analogous process we find the inverse
        \begin{align*}
            \hS^{-1} = \frac{1}{2} \pmqty{\admat[\sqrt{\mu_{+}} + \sqrt{\mu_{-}}]{-i\pqty{\sqrt{\mu_{+}} - \sqrt{\mu_{-}}} , i\pqty{\sqrt{\mu_{+}} - \sqrt{\mu_{-}}}}}
            \myperiod
        \end{align*}
        Now we can evaluate the transformation \eqref{eq:F-Q-similarity} and after some careful matrix multiplication we find 
        \begin{equation*}
            \hF = \hS \hH \hS^{-1}
            = \alpha E_0 \pmqty{\admat{1,1}}
            = \alpha E_0 \togglehat{\sigma}_x
            \mycomma
        \end{equation*}
        which is indeed Hermitian.
        
        
        \section{Deformation of the Harmonic Oscillator}