\chapter{\PT-Symmetric Hamiltonians}\label{ch:pt-symmetric-hamiltonians}
    \section{\PT-Symmetric Hamiltonians}
        Moving forward from what we said in \Sref{ch:pt-operators}, an operator \hA\ is said to be \PT-Symmetric if $\comm*{\hPT}{\hA} = 0$, which, in the case of the Hamiltonian \hH\ reads
        \begin{equation}
            \label{eq:PT-H-commutator}
            \comm*{\hPT}{\hH} = 0
            \myperiod
        \end{equation}
        The requirement of \PT-Symmetry is weaker than the one of Hermiticity as, from \PT-Symmetry alone, we can say little on the properties of the Hamiltonian's spectrum. As we found in \Sref{s:pt-symmetry}, the eigenvalues of a \PT-Symmetric Hamiltonian are real under the condition that \PT-Symmetry is not broken, meaning that the eneregy eigenvalues are real when the associated eigenstates are also simultaneous eigenstates for \hPT, which is not granted by \eqref{eq:PT-H-commutator} alone since \hPT\ is not a linear operator.

        Usually, when working with real quantum mechanical systems, the associated Hamiltonian is rarely Hermitian \cite{bender2024}. This is due to the fact that physical systems experience either loss to or gain from the environment---if not both at the same time. \PT-Symmetry on its end manages to describe systems that are coupled to the environment in such a way that gain and loss balance each other.

        Giving up on Hermiticity, however, comes at a price. As we will see in the next section, \PT-Symmetric Hamiltonians do not generally possess a complete set of eigenstates and the time evolution they generate is not unitary in the usual sense, meaning that probability may not be conserved when using the standard inner product
        \begin{equation}
            \label{eq:standard-inner-product}
            \braket{\phi}{\psi} = \int\nolimits_{\bbR^3} \phi^*\pqty{\vb{r}} \psi\pqty{\vb{r}} \dd[3]{\vb{r}}
            \myperiod
        \end{equation}
        This means that, while \PT-Symmetric Hamiltonians can have real eigenvalues, they cannot be used to build a physical theory of Quantum Mechanics unless we modify some of its foundations.

        %
        %
        %
        %
        %
        %  ================================================================
        %   AGGIUNGERE PREMESSA! Vedremo che la non rottura di simmetria è
        %   fondamentale... bisogna ridefinire il prodotto scalare in modo
        %   da avere ortonormalità... etc
        %  ================================================================
        %
        %
        %
        %
        %

    \section{The problem of completeness, unitarity and orthogonality}
        When workgin with \PT-Symmetric Hamiltonians, three problems arise:
        \begin{enumerate}[label = \mybullet]
            \item When \PT-Symmetry is broken, it is not always possible to find a complete set of eigenstates of the Hamiltonian for the Hilbert space.
            \item When \PT-Symmetry is broken, the energy spectrum becomes complex and the time evolution operator, which is built on the Hamiltonian as 
            \begin{equation}
                \label{eq:time-evolution}
                \hU\pqty{t} = \myexp{-i\frac{\hH}{\hbar}t}
                \mycomma
            \end{equation}
            will not preserve probability due to the introduction of exponentials of real numbers as coefficients.
            \item Even when \PT-Symmetry is unbroken and a complete set for the Hilbert space can be found, the set is not guaranteed to be orthogonal.
        \end{enumerate}
        To demonstrate what was just stated, let's look at a simple example built on a $2\times2$ matrix Hamiltonian \cite{Bender2005,Bender2007}\footnote{The presented Hamiltonian is derived from Bender's, by letting $s = 1$, $r = \gamma$ and $\theta = \pi/2$. The Hamiltonian implicitly has the dimensions of energy.}
        \begin{equation}
            \label{eq:ex:reduced-2x2-hamiltonian}
            \hH = \pmqty{\dmat[1]{i\gamma, -i\gamma}}
            \mycomma
        \end{equation}
        with $\gamma\in\bbR$ and the following definitions for \hP\ and \hT :
        \begin{equation}
            \label{eq:ex:reduced-2x2-P-T}
            \hP = \hat{\sigma}_x = \pmqty{\admat[0]{1,1}}
            \mycomma\qquad
            \hT = \hK
            \mycomma
        \end{equation}
        where \hK\ is complex conjugation. It is straightforward to see that $\hH^\dag \neq \hH$ and $\comm*{\hPT}{\hH} = 0$ for any value of $\gamma$ by using properties \eqref{eq:PT-commutator} and \eqref{eq:P-properties} and then evaluating $\hP\pqty*{\hK\hH\hK^{-1}}\hP = \hP\hH^*\hP$.
        
        The eigenvalues of \hH, obtained by solving $\det\pqty*{\hH - \lambda\hidM} = 0$, are
        \begin{equation*}
            \lambda_\pm = \pm\sqrt{1-\gamma^2}
        \end{equation*}
        which are real when $\abs{\gamma} < 1$, zero when $\abs{\gamma} = 1$ and complex when $\abs{\gamma} > 1$.
        
        Let's look at $\gamma = 1$ first. In this case, the Hamiltonian is still \PT-Symmetric and the eigenvalues are real and degenerate, however it is not possible to find two linearly independent eigenvectors for \hH. The only independent eigenvector is
        \begin{equation*}
            \ket{\phi_0} = \frac{1}{\sqrt{2}}\pmqty{1 \\ -i}
            \mycomma
        \end{equation*}
        which does not make for a complete set for $\bbC^2$.

        Let's now look at how breaking the \PT-Symmetry causes the loss of unitarity by looking at the case where $\gamma > 1$. We can express the eigenvalues as the two purely imaginary quantities
        \begin{equation*}
            \lambda_\pm = \pm i \sqrt{\gamma^2 - 1}
        \end{equation*}
        with eigenvectors
        \begin{equation*}
            \ket{\phi_\pm} = \pmqty{1 \\ -i\gamma \pm \lambda_\pm}
            \mycomma
        \end{equation*}
        which are not simultaneously eigenvectors for \hPT, since
        \begin{equation*}
            \hPT\ket{\phi_\pm} = \pqty{i\gamma + \lambda_\mp}\ket{\phi_\mp}
            \myperiod
        \end{equation*}
        We should note that the application of \hPT\ to either of the eigenvectors gives us a vector proportional to the other eigenvector. The time evolution operator is the function \eqref{eq:time-evolution} of \hH\ which, when applied to the eigenstates of \hH, will yield the same function of the corresponding eigenvalues:
        \begin{equation*}
            \hU\pqty{t}\ket{\phi_\pm} = \myexp{-i\frac{\hH}{\hbar}t}\ket{\phi_\pm} = \myexp{\mp\frac{\sqrt{\gamma^2-1}}{\hbar}t}\ket{\phi_\pm}
        \end{equation*}
        where the exponential actually diverges for $\ket{\phi_+}$ and goes to zero for $\ket{\phi_-}$. While we get two independent eigenvectors of \hH, consituting a complete set for $\bbC^2$, the time evolution operator that we built from the Hamiltonian does not preserve the norm and is not a unitary transformation.

        We can end this section by noting that when $\abs{\gamma} < 1$ and \PT-Symmetry is unbroken, the two eigenvectors we can find are not orthogonal according to the standard inner product. By letting $\alpha = \sqrt{1 - \gamma^2}$ we have 
        \begin{equation*}
            \ket{\phi_\pm} = \pmqty{1 \\ -i\gamma + \lambda_\pm} = \pmqty{1 \\ -i\gamma \pm \alpha}
            \myperiod
        \end{equation*}
        the inner product between the two eigenvectors is
        \begin{equation*}
            \bra{\phi_-}\ket{\phi_+} = \pmqty{1 & i\gamma - \alpha}\pmqty{1 \\ -i\gamma + \alpha} = 2 \gamma^2
            \mycomma
        \end{equation*}
        which is only zero when $\gamma = 0$.

    \section{Construction of the \hC\ operator and \CPT\ inner product}
    \section{Unitarity of \PT-Symmetric Quantum Mechanics}