
%=================%
%   PAGE LAYOUT   %
%=================%

\settypeblocksize{*}{1.2\lxvchars}{*} % lxvchars è una dimensione raccomandata che dipende dal font, utile per la leggibilità.
\setlrmargins{*}{*}{1}                % setta i margini in modo che siano 1:1 adeguandosi al typeblock.
\setulmarginsandblock{1.5in}{*}{1.4}  % setta i margini superiore e inferiore {superiore}{inferiore}{rapporto}.
\checkandfixthelayout                 % does the magic.



%=========================%
%   HEADER & PAGESTYLES   %
%=========================%

\setlength{\headwidth}{\textwidth}

\makepagestyle{thesis}                                                  % definisce il nome dello stile.
\makerunningwidth{thesis}{\headwidth}                                   % definisce la lunghezza dell'header.
\makeheadrule{thesis}{\headwidth}{\normalrulethickness}                 % mette la riga nell'header.
\makeheadposition{thesis}{flushright}{flushleft}{flushright}{flushleft} % definisce la posizione dell'header.
\makepsmarks{thesis}{%                                                  % da qui in poi non lo so
    \nouppercaseheads
    \createmark{chapter}{both}{shownumber}{\chaptername\ }{.\ } 
    \createmark{section}{right}{shownumber}{}{.\ }              
    \createplainmark{toc}{both}{\contentsname}                  
    \createplainmark{lof}{both}{\listfigurename}                
    \createplainmark{lot}{both}{\listtablename}
    \createplainmark{bib}{both}{\bibname}
    \createplainmark{index}{both}{\indexname}
    \createplainmark{glossary}{both}{\glossaryname}    
}
\makepagestyle{append}                                                  % definisce il nome dello stile.
\makerunningwidth{append}{\headwidth}                                   % definisce la lunghezza dell'header.
\makeheadrule{append}{\headwidth}{\normalrulethickness}                 % mette la riga nell'header.
\makeheadposition{append}{flushright}{flushleft}{flushright}{flushleft} % definisce la posizione dell'header.
\makepsmarks{append}{%                                                  % da qui in poi non lo so
    \nouppercaseheads
    \createmark{chapter}{both}{shownumber}{\appendixname\ }{.\ } 
    \createmark{section}{right}{shownumber}{}{.\ }              
    \createplainmark{toc}{both}{\contentsname}                  
    \createplainmark{lof}{both}{\listfigurename}                
    \createplainmark{lot}{both}{\listtablename}
    \createplainmark{bib}{both}{\bibname}
    \createplainmark{index}{both}{\indexname}
    \createplainmark{glossary}{both}{\glossaryname}    
}

%% se oneside:
    \makeoddhead{thesis}{\bfseries\leftmark}{}{\bfseries\thepage}
    \makeoddhead{append}{\bfseries\leftmark}{}{\bfseries\thepage}
%% se twoside:
    % \makeevenhead{thesis}{\bfseries\thepage}{}{\bfseries\leftmark}
    % \makeoddhead{thesis}{\bfseries\rightmark}{}{\bfseries}

%\chapterstyle{hangnum}
\aliaspagestyle{chapter}{empty} % toglie il numero di pagina nelle pagine con il nome del capitolo.
\aliaspagestyle{part}{empty}

\setsecnumdepth{subsection}